\chapter{Rappresentazione dei Numeri Macchina}

%//TODO recupera lezione 1

\section{Numeri in Virgola Mobile}

I calcolatori hanno registri di dimensioni finite e
ci sono quindi dei limiti per rappresentare i numeri reali.
Un problema fondamentale è associare ad ogni configurazione di cifre
binarie un numero reale in modo che l'insieme dei numeri rappresentati
sia distribuito correttamente nell'intervallo della retta reale.
Comunemente,er rappresentare i numeri reali, si usa una rappresentazione
\textit{a virgola mobile} o \textit{floating point}.
I numeri floating point si rappresentano in virgola mobile con una \textbf{mantissa} $f$ ed un esponente $e$, dove $\beta$, la base di rappresentazione è generalmente 10:

\begin{eqnarray*}
    f \beta^e & \text{con } 0 < f < 1
\end{eqnarray*}

\begin{defn}
    \textbf{Teorema di rappresentazione in base} \\
    Sia $x \in \R, x \neq 0$ e $\beta \geq 2$ una base di rappresentazione. Allora esistono e sono unici:
    \begin{enumerate}
        \item Un numero $p \in \Z$ detto esponente
        \item Una successione $\{d_i\}$ con $i \geq 1 \land 0 \leq d_i \leq \beta$ detta cifre della rappresentazione, tali che
    \end{enumerate}

    \begin{eqnarray*}
    & d_i \neq 0 & \\
    & \nexists t : d_i = \beta - 1 & \forall i \geq t \\
    & x = \sgn{x} \cdot \beta^p \sum_{i=1}^{\infty} d_i \beta^{-i}
    \end{eqnarray*}
    Dove $\sum_{i=1}^{+\infty}d_i \beta^{-i}$ è detta \textit{mantissa}.

    Se esiste un indice $k$ tale che $d_k \neq 0 \land d_i = 0 \forall i > k$ allora la rappresentazione si dice \textit{finita di lunghezza $k$}.
    Se $d_1 \neq 0$ allora la rappresentazione si dice \textit{normalizzata}.  

    Si osserva facilmente che
    \begin{eqnarray*}
    0.\overbar{9} = \sum_{i=1}^{\infty} 9 \cdot 10^{-i} = 1 \\
    =  9\sum_{i=1}^{\infty} 10^{-i} = 9 \left(\frac{1}{1-10^{-i} - 1}\right) = 1
    \end{eqnarray*}

    è una serie geometrica. Ricordiamoci la serie geometrica parziale:
    \begin{eqnarray*}
        S_n = \sum_{k=0}^{n} q^k = \frac{1-q^{n-1}}{1-q} & \text{se } q \neq 1 \\
        S_n = 1 & \text{se} q = 1
    \end{eqnarray*}
    %//TODO dimostrazione serie geometrica
    %//TODO per casa dimostra per induzione su n

\end{defn}

\begin{defn}
    \textbf{Insieme dei numeri di macchina} \\
    L'\textbf{Insieme dei Numeri di Macchina} è rappresentato da:
\begin{eqnarray*}
\F(\beta, t, m, M) = \{0\} \cup \{x \in \R : x = x = \sgn{x} \cdot \beta^p \sum_{i=1}^{t} d_i \beta^{-i} \}
\end{eqnarray*}
Laddove $\beta$ è la base, $t$ è il numero di cifre ed $m, M$ sono dei limiti per l'esponente tali che $-m \leq p \leq M$. La rappresentazione floating point impone che la rappresentazione sia normalizzata, quindi la prima cifra della mantissa deve essere diversa da 0: $d_1 \neq 0$. Le cifre devono essere in base, quindi $0 \leq d_i \leq \beta-1$. Poiché con $t$ cifre in base $\beta$ si possono rappresentare $\beta^t$ numeri, vanno esclusi quelli la cui prima cifra è nulla.


Se $x \neq 0$ ne si deduce facilmente che
$x \in \F \implies -x \in \F$.
Quanti sono però tutti i numeri rappresentabili nell'insieme dei numeri di macchina?
\begin{equation*}
    \begin{aligned}
        1 + 2(m + M + 1) (\beta -1) \beta^{t-1}
    \end{aligned}
\end{equation*}

\end{defn}
La prima configurazione è data dal numero 0, mentre le rimanenti sono date dalle possibili configurazioni dell'esponente e delle mantisse $d_i$.

\begin{defn}
    \textbf{$\omega$ e $\Omega$} \\
    Il numero più piccolo rappresentabile nell'insieme dei numeri di macchina sarà

    \begin{eqnarray*}
        \omega = \beta^{-1}(1 \cdot \beta^{-1} + 0 \cdot \beta^{-2} + \hdots + 0 \cdot \beta^{-t}) = \beta^{-m-1}
    \end{eqnarray*}

    mentre il numero più grande rappresentabile sarà

    \begin{eqnarray*}
        \Omega = \beta^M (\beta -1) \sum_{i=1}^{t} \beta^{-i} = \beta^M (1-\beta^{-t})
    \end{eqnarray*}
\end{defn}


\begin{defn}
    \textbf{Successore di un numero macchina} \\
    Il successore di un numero macchina $a \in \F(\beta, t, m, M) = \beta^p \cdot \sum_{i=1}^{t} d_i \beta^{-i}$ sarà $b = a + \beta^{p-t}$. Si ha anche che $\abs{b-a} = \beta^{p-t}$ è la distanza tra due numeri macchina successivi.

\end{defn}


\begin{defn}
    \textbf{Overflow e Underflow} \\
    Un numero reale non nullo $x = \pm 0.d_1d_2\hdots \beta^p$ appartiene a $\F$ soltanto se:
    
    \begin{itemize}
        \item $d_1 \neq 0$ (la rappresentazione è normalizzata)
        \item $-m \leq p \leq M$ (l'esponente appartiene all'intervallo degli esponenti rappresentabili) e $d_i = 0 \forall i > t$ (ogni cifra dopo il numero di cifre rappresentabili è nulla).
    \end{itemize}
    
    Se $x \notin \F(\beta, t, m, M)$ si pone il problema di \textbf{approssimare} $x$ ad un valore detto $\tilde{x}$ o segnalare un errore.
    Se l'esponente $p$ necessario per rappresentare $x$ non appartiene all'intervallo discreto dato dagli estremi $-m$ ed $M$ allora si verificano
    le condizioni di \textit{underflow} e \textit{overflow}. In tal caso un linguaggio di programmazione o un motore di calcolo numerico possono segnalare  un errore o restituire rispettivamente $\omega$ e $\Omega$. Se l'esponente $p$ appartiene all'intervallo degli esponenti rappresentabili in $\F$, ma le cifre $d_i$, con $i > t$ (non rappresentabili nei numeri macchina) non sono tutte nulle, ne consegue il bisogno di approssimare $x$ a $\tilde{x}$ con arrotondamento o troncamento.
\end{defn}


\begin{defn}
    \textbf{Standard IEEE-754 per numeri floating point a doppia precisione da 64 bit} \\ 
    Lo standard è il più usato dai costruttori di macchine per rappresenta numeri floating point. Lo standard accetta tre categorie di dati:
    \begin{enumerate}
        \item Numeri normalizzati: $d_1 \neq 0$
        \item Numeri non normalizzati
        \item Numeri speciali
    \end{enumerate}
    La base usata è $\beta = 2$ e ci sono tre possibili precisioni: semplice, doppia ed estesa. Vediamo la precisione doppia:

    In un registro da 64 bit sono contenuti:
    \begin{itemize}
        \item 1 bit di segno 
        \item 11 bit di esponente 
        \item 52 bit di mantissa
    \end{itemize}

    L'esponente $p$ è un intero che varia da -1024 a 1023, rappresentato in complemento a 2. Un valore di $p = 1023$ implica che l'esponente attuale è 0: $2^{p - 1023}$. Un valore dell'esponente di 1024 (tutti 1 nella configurazione di bit) serve per rappresentare numeri speciali. Se l'esponente è -1023 (tutti 0 nella configurazione), allora rappresenta dei numeri denormalizzati, compresi fra 0 ed $\omega$. Se si ha che
    un numero $x \notin (-\Omega, \Omega)$ allora si ha un caso di \textit{overflow}.
\end{defn}

\section{Arrotondamento, Troncamento, Errori assoluti e Relativi}

\begin{defn}
    \textbf{Arrotondamento e Troncamento} \\
    Dato un numero $\overbar{a} = 3.141592\hdots$ (proprio $\pi$),
    e una rappresentazione macchina con $t = 5, \F(10,5,m,M)$.
    Definiamo il troncamento come
    $\tilde{x} = \trn{\overbar{a}} = 3.1415$ e definiamo
    l'arrotondamento come $\tilde{x} = \arr{\overbar{a} = 3.1416}$.

    Più precisamente:
    \begin{eqnarray*}
        \trn{x} = \beta^p \sum_{i=1}^{t} d_i \beta^{-i} \\
        \arr{x} = \beta^p \trn{\sum_{i=1}^{t+1} d_i \beta^{-i} + \frac{1}{2} \beta^{-t}}
    \end{eqnarray*}

    Da notare che nell'arrotondamento di un numero si può verificare \textit{overflow}.
\end{defn}


    L'\textbf{errore assoluto} è  mentre l'errore relativo è
    \begin{eqnarray*}
        \text{errore assoluto} & \tilde{x} - x \\
        \text{errore relativo } & \epsilon_x = \frac{\tilde{x} - x}{x}
    \end{eqnarray*}


\begin{thm}
    Se non avviene overflow e underflow allora
    \begin{eqnarray*}
        \abs{\trn{x} - x} < \beta^{p-t}  \\
        \abs{\arr{x} - x} \leq \frac{1}{2} \beta^{p-t}
    \end{eqnarray*}

Dove la disuguaglianza vale $\iff d_{t+1} = \beta/2 \land d_{t+i} = 0 \forall i \geq 2$
\end{thm}

\begin{proof}
    Se $x = \Omega \implies \trn{x} - x = \arr{x} - x = 0$. Altrimenti, siano
    $a,b$ i due numeri di macchina consecutivi.

    \begin{eqnarray*}
        a = \beta^p \sum_{i=1}^{t} d_i \beta^{-i}, & & b = \beta^p \left( \sum_{i=1}^{t} d_i \beta^{-i} + \beta^{-i} \right)
    \end{eqnarray*}
    Avendo $a \leq x < b$ risulta che

    \begin{eqnarray*}
        \trn{x} = a & & x - \trn{x} < b - a = \beta^{p -t}
    \end{eqnarray*}

    Con la quale abbiamo dimostrato la prima disuguaglianza del teorema. Risulta poi che

    \begin{eqnarray*}
\arr{x} = \begin{cases}
    a & \text{se } x < \frac{a+b}{2} \\
    b & \text{se } x \geq \frac{a+b}{2}
\end{cases} \\
\implies \abs{\arr{x} - x} \leq \frac{b - a}{2} = \frac{1}{2} \beta^{p - t}
    \end{eqnarray*}

    L'uguaglianza per dimostrare la seconda disuguaglianza vale solo se

    \begin{eqnarray*}
        x = \frac{a + b}{2} \iff d_{t+1} = \frac{\beta}{2} \land d_{t+i} = 0 \forall i \geq 2
    \end{eqnarray*}
\end{proof}


\begin{thm}
    Sia $x \neq 0, x \in \R$, non in underflow o overflow e sia
    $\tilde{x} \in \F(\beta, t, m, M)$. Si ha allora che
    \begin{eqnarray*}
        \epsilon_x = \dfrac{\tilde{x} - x}{x} & \text{tale che} & \abs{\epsilon_x} < u \\
        & u = \begin{cases}
            \frac{1}{2} \beta^{1-t} & \text{se } \tilde{x} = \arr{x} \\
            \beta^{1-t} & \text{se } \tilde{x} = \trn{x} \\
        \end{cases}
    \end{eqnarray*}
    Dove $u$ è detta \textbf{precisione di macchina} o \textbf{epsilon} di macchina.
\end{thm}


    \begin{proof}
        % //TODO proof
        proof
    \end{proof}

    Si ha quindi,
    \begin{eqnarray*}
        \abs{\tilde{x} - x} < \abs{a -b } < \frac{1}{2} \abs{a - b}
        x = \beta^p\left(\sum_{i=1}^{+\infty} d_i \beta^{-i}\right) \\
        \tilde{x} = \beta^p \left( \sum_{i=1}^{t} c_i \beta^{-i} \right) \\
        \abs{x} \geq \beta^{p-1} \\
% \\TODO fix        \trn{x} = \tilde{x} \implies \abs{\dfrac{\tilde{x} - x}{x}} \leq \dfrac{\frac{1}{2} \beta^2}{2} \\
        \arr{x} = \tilde{x} \implies \abs{\dfrac{\tilde{x} - x}{x}} \leq \dfrac{\frac{1}{2} \beta^{p - t}}{x} < \frac{1}{2} \dfrac{\beta^{p-t}}{\beta^{p-1}} = \frac{1}{2} \beta^{1-t} \\
    \end{eqnarray*}


%% \\TODO fixa pag 16-19 bevilacqua