\chapter{Rappresentazione dei Numeri Macchina}

%//TODO recupera lezione 1

\section{Numeri in Virgola Mobile}

I calcolatori hanno registri di dimensioni finite e
ci sono quindi dei limiti per rappresentare i numeri reali.
Un problema fondamentale è associare ad ogni configurazione di cifre
binarie un numero reale in modo che l'insieme dei numeri rappresentati
sia distribuito correttamente nell'intervallo della retta reale.
Comunemente,er rappresentare i numeri reali, si usa una rappresentazione
\textit{a virgola mobile} o \textit{floating point}.
I numeri floating point si rappresentano in virgola mobile con una \textbf{mantissa} $f$ ed un esponente $e$, dove $\beta$, la base di rappresentazione è generalmente 10:

\begin{eqnarray*}
    f \beta^e & \text{con } 0 < f < 1
\end{eqnarray*}

\begin{defn}
    \textbf{Teorema di rappresentazione in base} \\
    Sia $x \in \R, x \neq 0$ e $\beta \geq 2$ una base di rappresentazione. Allora esistono e sono unici:
    \begin{enumerate}
        \item Un numero $p \in \Z$ detto esponente
        \item Una successione $\{d_i\}$ con $i \geq 1 \land 0 \leq d_i \leq \beta$ detta cifre della rappresentazione, tali che
    \end{enumerate}

    \begin{eqnarray*}
    & d_i \neq 0 & \\
    & \nexists t : d_i = \beta - 1 & \forall i \geq t \\
    & x = \sgn{x} \cdot \beta^p \sum_{i=1}^{\infty} d_i \beta^{-i}
    \end{eqnarray*}
    Dove $\sum_{i=1}^{+\infty}d_i \beta^{-i}$ è detta \textit{mantissa}.

    Se esiste un indice $k$ tale che $d_k \neq 0 \land d_i = 0 \forall i > k$ allora la rappresentazione si dice \textit{finita di lunghezza $k$}.
    Se $d_1 \neq 0$ allora la rappresentazione si dice \textit{normalizzata}.  

    Si osserva facilmente che
    \begin{eqnarray*}
    0.\overbar{9} = \sum_{i=1}^{\infty} 9 \cdot 10^{-i} = 1 \\
    =  9\sum_{i=1}^{\infty} 10^{-i} = 9 \left(\frac{1}{1-10^{-i} - 1}\right) = 1
    \end{eqnarray*}

    è una serie geometrica. Ricordiamoci la serie geometrica parziale:
    \begin{eqnarray*}
        S_n = \sum_{k=0}^{n} q^k = \frac{1-q^{n-1}}{1-q} & \text{se } q \neq 1 \\
        S_n = 1 & \text{se} q = 1
    \end{eqnarray*}
    %//TODO dimostrazione serie geometrica
    %//TODO per casa dimostra per induzione su n

\end{defn}

\begin{defn}
    \textbf{Insieme dei numeri di macchina} \\
    L'\textbf{Insieme dei Numeri di Macchina} è rappresentato da:
\begin{eqnarray*}
\F(\beta, t, m, M) = \{0\} \cup \{x \in \R : x = x = \sgn{x} \cdot \beta^p \sum_{i=1}^{t} d_i \beta^{-i} \}
\end{eqnarray*}
Laddove $\beta$ è la base, $t$ è il numero di cifre ed $m, M$ sono dei limiti per l'esponente tali che $-m \leq p \leq M$. La rappresentazione floating point impone che la rappresentazione sia normalizzata, quindi la prima cifra della mantissa deve essere diversa da 0: $d_1 \neq 0$. Le cifre devono essere in base, quindi $0 \leq d_i \leq \beta-1$. Poiché con $t$ cifre in base $\beta$ si possono rappresentare $\beta^t$ numeri, vanno esclusi quelli la cui prima cifra è nulla.


Se $x \neq 0$ ne si deduce facilmente che
$x \in \F \implies -x \in \F$.
Quanti sono però tutti i numeri rappresentabili nell'insieme dei numeri di macchina?
\begin{equation*}
    \begin{aligned}
        1 + 2(m + M + 1) (\beta -1) \beta^{t-1}
    \end{aligned}
\end{equation*}

La prima configurazione è data dal numero 0, mentre le rimanenti sono date dalle possibili configurazioni dell'esponente e delle mantisse $d_i$. Il numero più piccolo rappresentabile sarà $\omega = \beta^{-1}(1 \cdot \beta^{-1} + 0 \cdot \beta^{-2} + \hdots + 0 \cdot \beta^{-t}) = \beta^{-m-1}$ mentre il numero più grande rappresentabile sarà $\Omega = \beta^M (\beta -1) \sum_{i=1}^{t} \beta^{-i} = \beta^M (1-\beta^{-t})$. Il successore di un numero macchina $a \in \F(\beta, t, m, M) = \beta^p \cdot \sum_{i=1}^{t} d_i \beta^{-i}$ sarà $b = a + \beta^{p-t}$. Si ha anche che $\abs{b-a} = \beta^{p-t}$ è la distanza tra due numeri macchina successivi.

Un numero reale non nullo $x = \pm 0.d_1d_2\hdots \beta^p$ appartiene a $\F$ soltanto se: $d_1 \neq 0$ (la rappresentazione è normalizzata), $-m \leq p \leq M$ (l'esponente appartiene all'intervallo degli esponenti rappresentabili) e $d_i = 0 \forall i > t$ (ogni cifra dopo il numero di cifre rappresentabili è nulla).

Se $x \notin \F(\beta, t, m, M)$ si pone il problema di \textbf{approssimare} $x$ ad un valore detto $\tilde{x}$ o segnalare un errore.
Se l'esponente $p$ necessario per rappresentare $x$ non appartiene all'intervallo discreto dato dagli estremi $-m$ ed $M$ allora si verificano
le condizioni di \textit{underflow} e \textit{overflow}. In tal caso un linguaggio di programmazione o un motore di calcolo numerico possono segnalare  un errore o restituire rispettivamente $\omega$ e $\Omega$. Se l'esponente $p$ appartiene all'intervallo degli esponenti rappresentabili in $\F$, ma le cifre $d_i$, con $i > t$ (non rappresentabili nei numeri macchina) non sono tutte nulle, ne consegue il bisogno di approssimare $x$ a $\tilde{x}$ con arrotondamento o troncamento.

\end{defn}

%//TODO IEEE standard +inf e -inf
% 1 bit di segno, 11 di esponente e 52 di mantissa (1 double). p va da -1021 a 1024. se gli esponenti sono tutti ad 1 allora se m = 0 allora +-inf. se m \neq 0 allora nan (configurazione speciale).
% Se gli esponenti sono tutti 0 allora rappresenta dei numeri denormalizzati (fra 0 e \omega) che si scrivono con (0.0 d_2, d_3, d_4)
% Se un numero $x \notin (-\Omega, \Omega)$ allora va in overflow

%se $\abs{x} < \omega$ \implies \text{underflow} \implies 0
%//TODO arrotondamento, ERRORE ASSOLUTO E RELATIVO

\begin{defn}
    \textbf{Arrotondamento, Troncamento, Errori assoluti e Relativi} \\
    Dato un numero $\overbar{a} = 3.141592\hdots$ (proprio $\pi$)
    Nella rappresentazione macchina abbiamo $t = 5, \F(10,5,m,M)$.
    Definiamo il troncamento come
    $\tilde{x} = \trn{\overbar{a}} = 3.1415$ e definiamo
    l'arrotondamento come $\tilde{x} = \arr{\overbar{a} = 3.1416}$.

    L'\textbf{errore assoluto} è $\tilde{x} - x$ mentre l'errore relativo è $\frac{\tilde{x} - x}{x}$
\end{defn}


\begin{lem}
    Se non avviene overflow e underflow allora
    \begin{eqnarray*}
        \abs{\tilde{x} - x} < \beta^{p-t} & \text{se } \tilde{x} = \trn{x} \\
        \abs{\tilde{x} - x} \leq \frac{1}{2} \beta^{p-t} & \text{se } \tilde{x} = \arr{x}
    \end{eqnarray*}
\end{lem}

\begin{defn}
    \textbf{Teorema dell'errore relativo} \\
    
\end{defn}
